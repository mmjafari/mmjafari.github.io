% Created 2025-07-04 Fri 15:31
% Intended LaTeX compiler: xelatex
\documentclass[presentation]{beamer}
\usepackage{graphicx}
\usepackage{longtable}
\usepackage{wrapfig}
\usepackage{rotating}
\usepackage[normalem]{ulem}
\usepackage{capt-of}
\usepackage{hyperref}
\usepackage{amsmath}
\usepackage{amsfonts}
\usepackage{braket}
\DeclareFontFamily{U}{wncy}{}
\DeclareFontShape{U}{wncy}{m}{n}{<->wncyr10}{}
\DeclareSymbolFont{mcy}{U}{wncy}{m}{n}
\DeclareMathSymbol{\Sh}{\mathord}{mcy}{"58}
\usetheme{Copenhagen}
\usecolortheme{default}
\author{Jafari M.M.}
\date{\today}
\title{On Selmer Ranks of Elliptic Curves With Rational 2-Torsion}
\hypersetup{
 pdfauthor={Jafari M.M.},
 pdftitle={On Selmer Ranks of Elliptic Curves With Rational 2-Torsion},
 pdfkeywords={},
 pdfsubject={},
 pdfcreator={Emacs 31.0.50 (Org mode 9.7.11)}, 
 pdflang={English}}
\begin{document}

\maketitle
\begin{frame}{Outline}
\tableofcontents
\end{frame}

\section{Preliminaries}
\label{sec:orgc854796}

\begin{frame}[label={sec:orgc05925c}]{Hillbert's 10th Problem}
\begin{block}{Question}
Given a number field \(k\), is there any algorithm to determine whether a
polynomial \(f \in \mathbb{Q}[x_1,\dots,x_n]\) admits a solution in \(k\)?
\end{block}

\begin{block}{Answer}
\(k = \mathbb{Z} \to\) No. [Davis-Putnam-Robinson + Matiyasevich]

\(k = \mathbb{Q} \to\) Yes in one variable, unknown in general.
\end{block}
\end{frame}
\begin{frame}[label={sec:org9e155ca}]{Mordell-Weil Theorem}
\begin{theorem}[Mordell-Weil-Faltings]
Given a variety \(C\) of genus \(g\), the set \(C(\mathbb{Q})\) is determined as

\(g = 0 \to 0, \infty\)

\(g = 1 \to G \times \mathbb{Z}^r\)

\(g \ge 2 \to\) finite
\end{theorem}

We now focus on the case \(g = 1\).
\end{frame}
\begin{frame}[label={sec:orgb900c89}]{Elliptic Curves}
\begin{definition}
An \emph{elliptic curve} \(E\) over a number field \(k\) is a smooth, projective
algebraic variety which in \(Char(k) \ne 2,3\) can be expressed as
$$y^2 = x^3 + Ax + B$$
\end{definition}

It follows from the theorem above that
\(E/\mathbb{Q} \simeq E_{tors} \times \mathbb{Z}^r\)
\end{frame}
\begin{frame}[label={sec:org239ecfc}]{Selmer groups}
Given two elliptic curves \(E,E'\) and isogenies \(\varphi,\varphi'\) we have
$$0 \longrightarrow E[\varphi] \longrightarrow E \longrightarrow E'
\longrightarrow 0$$
which produces the exact sequence of Galois cohomology
$$0 \longrightarrow E'(k)/\varphi(E(k))
\overset{\delta_k}{\longrightarrow} H^1(k,E[\varphi]) \longrightarrow
H^1(k,E)[\varphi] \longrightarrow 0$$
where \(\delta_k\) is the connecting homomorphism.

For each \(\nu\)-adic completion of \(k\) we define the following groups
$$Sel^\varphi(E/k) = Ker\{H^1(k,E[\varphi]) \to \prod_\nu H^1(k_\nu, E)[\varphi]\}$$
$$\Sh(E/k) = Ker\{H^1(k,E) \to \prod_\nu H^1(k_\nu, E)\}$$
\end{frame}
\begin{frame}[label={sec:org6a5e27d}]{Selmer groups}
These two groups form the short exact sequence
$$0 \longrightarrow E'(\mathbb{Q})/\varphi(E(\mathbb{Q}))
\longrightarrow Sel^\varphi(E/\mathbb{Q}) \longrightarrow
\Sh(E/\mathbb{Q})[\varphi] \longrightarrow 0$$

It holds that
$$rank(E/\mathbb{Q}) \le dim_{\mathbb{F}_2}Sel^\varphi(E/\mathbb{Q}) +
dim_{\mathbb{F}_2}Sel^{\varphi'}(E'/\matnbb{Q}) - 2$$
\end{frame}
\begin{frame}[label={sec:org3528283}]{Tamagawa ratios}
\begin{definition}[Tamagawa ratio]
The ratio
$$\mathscr{T}(E/E') = \frac{|Sel^\varphi(E/K)|}{|Sel^{\varphi'}(E'/K)|}$$
is called the \emph{Tamagawa ratio} associated to isogenues curves.
\end{definition}

In [KLO13,KLO14] the distribution of these ratios in isogenous families
and quadratic twists is used to study the Selmer ranks.
\end{frame}
\section{2-Torsion families}
\label{sec:orgdc821d8}

\begin{frame}[label={sec:org8a307dc}]{Elliptic Curves with a 2-Torsion}
Consider a family of curves \(E_r\) with a given 2-torsion point \((r,0)\). Such family
can be parametrized as
$$E_r : y^2 = x^3 + tx - rt -r^3$$
which after a translation \((r,0) \mapsto (0,0)\) becomes
$$E : y^2 = x^3 + 2rx^2 + (r^2+t)x$$
equipped with an isogenous curve
$$E' : y^2 = x^3 - 6rx^2 - (3r^2+4t)x$$
\end{frame}
\begin{frame}[label={sec:org7ecad74}]{Elliptic Curves with a 2-Torsion}
Certain results are known about the average ranks of such families; in
particular, we have the following

\begin{theorem}[Klagsburn-Lemke Oliver]
Given a family of elliptic curves with a 2-torsion, the rank \(Sel_2(E/\mathbb{Q})\)
grows with respect to the height of the family (e. g, the average is not a
constant).
\end{theorem}

We would like to examine further the properties of Selmer group given \(r,t\).
\end{frame}
\begin{frame}[label={sec:org4923200}]{Connecting Homomorphisms and Selmer Groups}
in [G02], two alogorithms are given for calculation of connecting
homomorphisms \(\delta_p\) and \(\delta_2\). These images are used coupled with the definition
\begin{multline*}
\(S^\varphi(E/\mathbb{Q}) = \{x \in H^1(\mathbb{Q},E[\varphi])\  | \ res_p(x) \in Im(\delta_p)\
\text{for all places }p\} \\ = \bigcap Im(\delta_p)\)
\end{multline*}
to describe the full Selmer group.
\end{frame}
\begin{frame}[label={sec:org6b2c48a}]{Algorithms for \(\delta_p\)}
In an elliptic curve \(y^2 = x^3 + Ax^2 + Bx\), let
\(a = ord_p(A), b = ord_p(B), d = ord_p(A^2-4B)\).

The algorithms deal with \(b = 0,1,2,3\) for all \(p\), fully
reproducing the Selmer groups.

We observe that in case \(b \ge 1\) and \(a \ge 2\), we have
\(E : y = x^3 + pA'x + p^2B'x\) is a quadratic twist \(E^p\), thus
the algorithm above can recursively construct the Selmer
group for higher cases omitted in the description.
\end{frame}
\section{Our work}
\label{sec:org398e269}

\begin{frame}[label={sec:org375f1cc}]{Statistical Arguements for the Selmer ranks}
Since the algorithm roughly implies a direct relation
between upper bound of Selmer ranks and the order \(b\), we can deduce the
following upper bound for the \(p\) part of the image
$$sup(\delta_p) \sim \sum_{n = 1}^\infty \frac{n}{p^{n}} = \frac{p}{(p-1)^2}$$

Adding all the contributions of primes \(p\) up to the naive height
\(X\) of the curve family we get
$$sup(r) \sim \sum_{p \text{ prime}}^X \frac{p}{(p-1)^2} \sim \sum_p^X \frac{1}{p} \sim \log(\log(x))$$
\end{frame}
\begin{frame}[label={sec:orge060821}]{Statistical Arguments for the Selmer ranks}
The observation above agrees with [KLO13] and [KK17], in which the distribution
for Selmer ranks of quadratic twists is observed to fall in the
normal distribution \(\mathscr{N}(0, \frac{1}{2}\log\log X)\), following different methods.
\end{frame}
\begin{frame}[label={sec:org4f83c18}]{Other implications}
Given the direct relation of orders, for the curve with 2-torsion
at \((r,0)\) we expect the highest relation with the order of \(2\) in \(r\),
as \(ord_2(A) \ge 1\) is most likely to result in non-trivial Selmer group.
We can observe this by taking the average rank for \(0<r \le 100\) and
\(0 < t \le 10,000\), plotted in the following figure
\end{frame}
\begin{frame}[label={sec:orga2044a4}]{Plot for \(r\)}
\begin{center}
\includegraphics[width=200]{/home/mmj/tests/leplot.png}
\end{center}
\end{frame}
\begin{frame}[label={sec:orgf88e266}]{Citations}
KLO13: THE DISTRIBUTION OF 2-SELMER RANKS OF QUADRATIC TWISTS
OF ELLIPTIC CURVES WITH PARTIAL TWO-TORSION, Z. Klagsburn, R. Lemke Oliver

KLO14: THE DISTRIBUTION OF THE TAMAGAWA RATIO IN THE FAMILY
OF ELLIPTIC CURVES WITH A TWO-TORSION POINT, Z. Klagsburn, R. Lemke Oliver

G02: A STUDY ON THE SELMER GROUPS OF ELLIPTIC CURVES
WITH A RATIONAL 2-TORSION, T. Goto

KK17: ON THE JOINT DISTRIBUTION OF \(Sel_\varphi(E/Q)\) AND \(Sel_{\hat{\varphi}} (E′/Q)\) IN
QUADRATIC TWIST FAMILIES, D. Kane, Z. Klagsburn
\end{frame}
\end{document}
